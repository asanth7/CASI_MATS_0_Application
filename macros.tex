\renewcommand{\epsilon}{\varepsilon} % (this paper is sane-certified)
\newcommand{\TODO}{{\color{red} [TODO]}}
\newcommand{\CITE}{{\color{red} [CITE]}}
\newcommand{\matan}[1]{{\color{purple} m: #1}}
\newcommand{\E}{\mathop{\mathbb{E}}}

%%%%%%%%%%%%%%%%%%%%%%%%%%%%%%%%%%%%%%%%%%%%%%%%%%%%%%%%
% begin region MATAN MACROS FILE
% \input{lib/matan_includes} %paths are relative to main.tex! 
\input{lib/modifier-shortcuts}
% \input{lib/snark-shortcuts}
\input{lib/paper-specific}


%%%%%%%%%%%%%%%%%%%%%%%%%%%%%%%%%%%%%%%%%%%%%%%%%%%%%%%%%%%%%
%%%%%%%%%%%%%%%%%%%%%%%%%%%%%%%%%%%%%%%%%%%%%%%%%%%%%%%%%%%%%
% Envieroments

\newmdenv[
    topline=false,
    bottomline=false,
    rightline=false,
    leftline=true, % Only the left line is enabled
    linecolor=gray!40, % Light gray color for the line
    linewidth=4pt, % Width of the line
    leftmargin=5pt, % Indent from the left margin
    rightmargin=10pt, % Right margin spacing
    innertopmargin=5pt, % Space above the text inside the frame
    innerbottommargin=5pt % Space below the text inside the frame
]{indentedtext}

%! status does not play nice with flot yet
\newmdenv[
    topline=false,
    bottomline=true,
    rightline=false,
    leftline=true, % Only the left line is enabled
    linecolor=purple!40, % Light purple color for the line
    linewidth=4pt, % Width of the line
    leftmargin=5pt, % Indent from the left margin
    rightmargin=10pt, % Right margin spacing
    innertopmargin=5pt, % Space above the text inside the frame
    innerbottommargin=10pt, % Space below the text inside the frame
    roundcorner=10pt, % Rounded corners
]{statusReportInternal}

\newcommand{\status}[2]{
    \begin{statusReportInternal}
        \parhead{Status}
        \emph{#1}\\
        \par\noindent #2
    \end{statusReportInternal}
}

\newenvironment{out}
{\begin{outline}[enumerate]
}
{ 
    \end{outline}
}

\newenvironment{researchquestion}
{
    \itshape
    \begin{center}
    \vspace{.1em}
}
{
    \end{center}
    \vspace{.1em}
}

\NewDocumentCommand{\largeProb}{m m m O{=}}{
    \begin{equation*}
        \Pr\insqr{#1 \;\left| \;  
        \begin{aligned}
            #2 
        \end{aligned}
        \right.} #4 #3
    \end{equation*}
    }
    
\newcommand{\diffLargeProb}[5]{
\begin{dmath*}
\inabs{
    \Pr \insqr{ #1 \; \left| \;
    \begin{aligned}
        #2
    \end{aligned}
    }\right.
    -
    \Pr \insqr{ #3 \; \left| \;
    \begin{aligned}
        #4
    \end{aligned}
    }\right.
    } = #5	
\end{dmath*}
}

\newcommand{\eqLargeDist}[5]{
    \begin{dmath*}
        \inset{ #1 \; \left| \;
        \begin{aligned}
            #2
        \end{aligned}
        }\right.
        #3
        \inset{ #4 \; \left| \;
        \begin{aligned}
            #5
        \end{aligned}
        }\right.	
    \end{dmath*}
    }

%%%%%%%%%%%%%%%%%%%%%%%%%%%%%%%%%%%%%%%%%%%%%%%%%%%%%%%%%%%%%
%%%%%%%%%%%%%%%%%%%%%%%%%%%%%%%%%%%%%%%%%%%%%%%%%%%%%%%%%%%%%
%%%% in*  
\newcommand{\inpar}[1]{\ensuremath{\left( #1 \right)}\xspace}
\newcommand{\insqr}[1]{\ensuremath{\left[ #1 \right]}\xspace}
\newcommand{\inabs}[1]{\ensuremath{\left| #1 \right|}\xspace}
\newcommand{\inset}[1]{\ensuremath{\left\{ #1 \right\}}\xspace}
\newcommand{\inceil}[1]{\ensuremath{\left\lceil #1 \right\rceil}\xspace}
\newcommand{\infloor}[1]{\ensuremath{\left\lfloor #1 \right\rfloor}\xspace}


%%%%%%%%%%%%%%%%%%%%%%%%%%%%%%%%%%%%%%%%%%%%%%%%%%%%%%%%%%%%%
%%%%%%%%%%%%%%%%%%%%%%%%%%%%%%%%%%%%%%%%%%%%%%%%%%%%%%%%%%%%%
% General Math 

%markdown
\newcommand{\unestablished}{\XXX[unestablished]}


%symbols
\renewcommand{\phi}{\varphi}
\newcommand{\eps}{\ensuremath{\epsilon}\xspace}
\renewcommand{\emptyset}{\varnothing}
\newcommand{\Z}{\ensuremath{\mathbb{Z}}}
\newcommand{\ZZ}[1]{\ensuremath{\Z_{#1}}}
\newcommand{\vv}[1]{\ensuremath{\vec{#1}}\xspace}
\newcommand{\qeq}{{\stackrel{?}{=}}}
\newcommand{\tensor}{\otimes}
%
\newcommand{\GeneratedBy}[1]{\ensuremath{\langle #1 \rangle}\xspace}
\newcommand{\convolution}{\ensuremath{*}\xspace}
\newcommand{\divides}{\ensuremath{\:|\:}\xspace}

%decorators 
\newcommand{\inv}{\ensuremath{{^{-1}}}\xspace}
\renewcommand{\sc}[1]{^{(#1)}}
\newcommand{\Support}{{\mathrm{Support}}}
\newcommand{\norm}[1]{\left\lVert#1\right\rVert}


%% graph theory
\newcommand{\Graph}{G}
\newcommand{\Nei}{\ensuremath{\mathrm{Nbr}}}
\newcommand{\eigen}{\ensuremath{\lambda}\xspace}
\newcommand{\Cayley}[2]{\mathrm{Cay}(#1, #2)}
\newcommand{\CayleySquared}[3]{\mathrm{Cay}^2(#1, #2, #3)}


%indexes
\newcommand{\istar}{\ensuremath{i^\star}\xspace}
\newcommand{\jstar}{\ensuremath{j^\star}\xspace}

%styling
\newcommand{\Greymidrule}{\arrayrulecolor{lightgray}\midrule\arrayrulecolor{black}}
\newcommand{\horizLine}{\noindent\textcolor[RGB]{220,220,220}{\rule{\linewidth}{2pt}}}
\newcommand{\suchthat}{\ensuremath{~\middle|~}}
\newcommand{\defined}{\ensuremath{:=}}

% Lin Algebra 
\newcommand{\spanLA}{\ensuremath{\mathrm{span}}\xspace}


% other math 
\newcommand{\perm}{\pi}


